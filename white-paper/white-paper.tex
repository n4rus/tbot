\documentclass[a4paper,12pt]{article}
\usepackage[T1]{fontenc}
\usepackage[utf8]{inputenc}
\usepackage{datetime}
\usepackage{hyperref}

\hypersetup{
    colorlinks=true,
    linkcolor=black,
    filecolor=black,      
    urlcolor=black,
    pdftitle={Sharelatex Example},
    bookmarks=true,
    pdfpagemode=FullScreen,
}

%document-type-curriculum_vitae
%the blank fields, and descriptions must be substituted by user's information 
\begin{document}
\pagestyle{empty}
\begin{center}
  \huge
  \vspace*{-5cm}
  \textbf{Trade Robot(tbot)}
  \vspace*{.015\baselineskip}
  %\linebreak
  \end{center}
\begin{center}
\begin{flushleft}
   Trade Robot or 'tbot' for short is a program that gets data from the api
  provided by novadax in default mode, makes dataframes, handle trades in
  auto loop, and gives output in xlsx at the end.
  \linebreak
  \vspace*{0.05\baselineskip}
   The same algoritm may work to S\&P 500 as well, and under few adjustements
  can be moded into it. It was first release under 'gpl-3.0' license, meaning
  it was intend to be an open-source software.
  \linebreak
  \vspace*{0.05\baselineskip}
  As final product from the trades handling, liquidity is gathereded in a
  minute/second to daily movement. Science is meant to excel human development.
  \linebreak
  \vspace*{0.05\baselineskip}
  This code is free software no one shall charge you for it. However if you
  want to contribute with the developer, here is the support-wallet:
  \linebreak
  \vspace*{0.05\baselineskip}
  btc: 1HWaLksrPGCK8bGyzRVPQXCcrFgc9xh1oZ
  \linebreak
  \vspace*{0.05\baselineskip}
  eth: 0x3bea034bbba1d84c505c38dce8d935279e9b5337
  \linebreak
  \vspace*{0.05\baselineskip}
   The project was released on version 0.9.9.7 and it may be countinued if
  it proves to be effective. In the book "Intriguing-surreality", the server
  was planned to have solar or wind power since we the computing power could
  start at mobile motherboards and laptops. At that point user may share
  their own exceeding energy per square meter wich resumes to:
  \vspace*{0.05\baselineskip}
  \begin{equation}
    Earth = \sum * User(i)
    \end{equation}
  \vspace*{0.05\baselineskip}
  User = i
  \vspace*{0.05\baselineskip}
  \begin{center}
    \begin{equation}
  \sum{i}  (\int_t^t \sum n * \frac{E}{m^2})
  \vspace*{0.05\baselineskip}
    \end{equation}
  \end{center}
    The equation represents a tie between the user, and the area of gathering.
    Given any kind of tech, we can apply the same measures to compare the most
    efficient one, and put it to function there. Always who have more land can
    profit more, but it may depend on how to use it, than this become a real
    state market that depends on renewable energy deployment, as well its
    constant research and development within the blockchain that is already a
    real piece of land that requires enourmous amouts of energy.
    \linebreak
    \vspace*{0.05\baselineskip}
    For managing that amount of energy new ways of making big data and cloud
    computing shall come as the progression of technology goes on. So the code
    and the methods of doing it must do it within. The constant need for
    new sources of energy and better management of it is an well know fact,
    even if generation of solar energy in houses can be tricky there is no
    meaning in not pursuing a renewable system in our moder world.
\end{flushleft}
\end{center}
\begin{center}
    \today
  \end{center}
\end{document}
